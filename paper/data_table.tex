[?1034hPython 2.7.9 (default, Mar 30 2015, 11:26:35) 
Type "copyright", "credits" or "license" for more information.

IPython 3.1.0 -- An enhanced Interactive Python.
?         -> Introduction and overview of IPython's features.
%quickref -> Quick reference.
help      -> Python's own help system.
object?   -> Details about 'object', use 'object??' for extra details.

In [1]: 
In [2]: 
In [3]: 
In [3]:    ...:    ...:    ...:    ...:    ...:    ...:    ...:    ...:    ...:    ...:    ...:    ...:    ...:    ...:    ...:    ...:    ...:    ...:    ...:    ...:    ...:    ...:    ...:    ...:    ...:    ...: 
In [4]: \documentclass[10pt,a4]{article}
% better looking tables with `\toprule`,`\midrule`,`\bottomrule`:
\usepackage{booktabs}
\begin{document}
  \begin{table}[h!]
    \begin{center}
      \begin{tabular}{lcccc}
        \toprule
        \textbf{Variant} & \textbf{Num Sequences} & \textbf{Num Features} & \textbf{Taxonomic Span} \\
        H3 & & & 
        canonicalH3 & \begin{tabular}{@{}c@{}}Curated set: 15 \\Automically extracted set: 2069\end{tabular} & 0 & Eukaryotes \\
        cenH3 & \begin{tabular}{@{}c@{}}Curated set: 12 \\Automically extracted set: 418\end{tabular} & 0 & Eukaryotes \\
        H3.3 & \begin{tabular}{@{}c@{}}Curated set: 15 \\Automically extracted set: 549\end{tabular} & 0 & Eukaryotes \\
        H3.5 & \begin{tabular}{@{}c@{}}Curated set: 6 \\Automically extracted set: 1246\end{tabular} & 0 & Hominids \\
        H3.Y & \begin{tabular}{@{}c@{}}Curated set: 8 \\Automically extracted set: 78\end{tabular} & 0 & Primates \\
        TS_H3.4 & \begin{tabular}{@{}c@{}}Curated set: 8 \\Automically extracted set: 2146\end{tabular} & 0 & mammals \\
        H4 & & & 
        canonicalH4 & \begin{tabular}{@{}c@{}}Curated set: 14 \\Automically extracted set: 4327\end{tabular} & 0 & Eukaryotes \\
        H2A & & & 
        canonicalH2A & \begin{tabular}{@{}c@{}}Curated set: 23 \\Automically extracted set: 4081\end{tabular} & 0 & Eukaryotes \\
        H2A.B & \begin{tabular}{@{}c@{}}Curated set: 15 \\Automically extracted set: 139\end{tabular} & 0 & Mammals \\
        H2A.L & \begin{tabular}{@{}c@{}}Curated set: 17 \\Automically extracted set: 186\end{tabular} & 0 & Certain mammals \\
        H2A.M & \begin{tabular}{@{}c@{}}Curated set: 11 \\Automically extracted set: 95\end{tabular} & 0 & Mammals \\
        H2A.W & \begin{tabular}{@{}c@{}}Curated set: 12 \\Automically extracted set: 869\end{tabular} & 0 & Plants \\
        H2A.X & \begin{tabular}{@{}c@{}}Curated set: 22 \\Automically extracted set: 1149\end{tabular} & 0 & Eukaryotes except nematode \\
        H2A.Z & \begin{tabular}{@{}c@{}}Curated set: 26 \\Automically extracted set: 2605\end{tabular} & 0 & Eukaryotes \\
        macroH2A & \begin{tabular}{@{}c@{}}Curated set: 10 \\Automically extracted set: 1216\end{tabular} & 0 & Vertebrates \\
        TS_H2A.1 & \begin{tabular}{@{}c@{}}Curated set: 15 \\Automically extracted set: 859\end{tabular} & 0 & Mammals \\
        H2B & & & 
        canonicalH2B & \begin{tabular}{@{}c@{}}Curated set: 26 \\Automically extracted set: 6528\end{tabular} & 0 & Eukaryotes \\
        H2B.W & \begin{tabular}{@{}c@{}}Curated set: 6 \\Automically extracted set: 245\end{tabular} & 0 & Mammals \\
        H2B.Z & \begin{tabular}{@{}c@{}}Curated set: 3 \\Automically extracted set: 212\end{tabular} & 0 & Apicomplexa \\
        sperm_H2B & \begin{tabular}{@{}c@{}}Curated set: 6 \\Automically extracted set: 97\end{tabular} & 0 & Echinacea(?) \\
        subH2B & \begin{tabular}{@{}c@{}}Curated set: 11 \\Automically extracted set: 86\end{tabular} & 0 & Primates, rodents, marsupials, and bovids \\
        TS_H2B.1 & \begin{tabular}{@{}c@{}}Curated set: 5 \\Automically extracted set: 456\end{tabular} & 0 & Mammals \\
        H1 & & & 
        genericH1 & \begin{tabular}{@{}c@{}}Curated set: 13 \\Automically extracted set: 4341\end{tabular} & 0 & Eukaryotes \\
        H1.0 & \begin{tabular}{@{}c@{}}Curated set: 16 \\Automically extracted set: 602\end{tabular} & 0 & Metazoa \\
        H1.10 & \begin{tabular}{@{}c@{}}Curated set: 6 \\Automically extracted set: 146\end{tabular} & 0 & Vertebrates \\
        OO_H1.8 & \begin{tabular}{@{}c@{}}Curated set: 2 \\Automically extracted set: 250\end{tabular} & 0 & Mammals \\
        ScH1 & \begin{tabular}{@{}c@{}}Curated set: 2 \\Automically extracted set: 443\end{tabular} & 0 & Saccharomyces(?) \\
        TS_H1.6 & \begin{tabular}{@{}c@{}}Curated set: 13 \\Automically extracted set: 493\end{tabular} & 0 & Mammals \\
        TS_H1.7 & \begin{tabular}{@{}c@{}}Curated set: 2 \\Automically extracted set: 144\end{tabular} & 0 & Mammals \\
        TS_H1.9 & \begin{tabular}{@{}c@{}}Curated set: 4 \\Automically extracted set: 101\end{tabular} & 0 & Mammals \\
        Unknown & & & 
        \bottomrule
      \end{tabular}
    \end{center}
  \end{table}
\end{document}

In [5]: 
Do you really want to exit ([y]/n)? 
