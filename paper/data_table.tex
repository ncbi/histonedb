\documentclass[a4paper,landscape]{article}
    \usepackage[landscape]{geometry}
% better looking tables with `\toprule`,`\midrule`,`\bottomrule`:
\usepackage{booktabs}
\pagenumbering{gobble}
\begin{document}
\begin{table}[h!]
\begin{center}
\begin{tabular}{lccccc}
\toprule
 & \multicolumn{2}{c}{\textbf{Number of Sequences} } &  & \\
\textbf{Variant}&\textbf{Curated} &\textbf{Automatic} &\textbf{Number of Features} &  \textbf{Taxonomic Span} \\
\toprule 
        canonicalH3 & 15 &2069  & 11 & Eukaryotes \\
        cenH3 & 12 &418  & 14 & Eukaryotes \\
        H3.3 & 15 &549  & 12 & Eukaryotes \\
        H3.5 & 6 &1246  & 11 & Hominids \\
        H3.Y & 8 &78  & 12 & Primates \\
        TS H3.4 & 8 &2146  & 11 & mammals \\
\toprule 
        canonicalH4 & 14 &4327  & 10 & Eukaryotes \\
\toprule 
        canonicalH2A & 23 &4081  & 17 & Eukaryotes \\
        H2A.B & 15 &139  & 21 & Mammals \\
        H2A.L & 17 &186  & 17 & Certain mammals \\
        H2A.M & 11 &95  & 17 & Mammals \\
        H2A.W & 12 &869  & 19 & Plants \\
        H2A.X & 22 &1149  & 19 & Eukaryotes except nematode \\
        H2A.Z & 26 &2605  & 20 & Eukaryotes \\
        macroH2A & 10 &1216  & 19 & Vertebrates \\
        TS H2A.1 & 15 &859  & 17 & Mammals \\
\toprule 
        canonicalH2B & 26 &6528  & 9 & Eukaryotes \\
        H2B.W & 6 &245  & 10 & Mammals \\
        H2B.Z & 3 &212  & 9 & Apicomplexa \\
        sperm H2B & 6 &97  & 9 & Echinacea(?) \\
        subH2B & 11 &86  & 11 & Primates, rodents, marsupials, bovids \\
        TS H2B.1 & 5 &456  & 9 & Mammals \\
\toprule 
        genericH1 & 13 &4341  & 7 & Eukaryotes \\
        H1.0 & 16 &602  & 7 & Metazoa \\
        H1.10 & 6 &146  & 7 & Vertebrates \\
        OO H1.8 & 2 &250  & 7 & Mammals \\
        ScH1 & 2 &443  & 13 & Saccharomyces(?) \\
        TS H1.6 & 13 &493  & 7 & Mammals \\
        TS H1.7 & 2 &144  & 7 & Mammals \\
        TS H1.9 & 4 &101  & 7 & Mammals \\
\toprule 
\bottomrule
\end{tabular}
\end{center}
\end{table}
\end{document}
